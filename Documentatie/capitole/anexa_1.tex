\begin{center}
	\textbf{Anexa numarul 1}
\end{center}

\textbf{Configurația algoritmului genetic pentru primul experiment este următoarea: }

- numărul de runde jucate în fiecare meci: 10;\\
- numărul de generații: 100;\\
- dimensiunea populației antrenate: 10;\\
- probabilitatea de încrucișare: 0.1;\\
- probabilitatea de mutație: 0.1;\\
- configurația populației de antrenament:\\
\begin{center}
	\begin{lstlisting}
	{
		"Always Cooperate": 1,
		"Always Defect": 1,
		"Grudger": 1,
		"Pavlov": 1,
		"Tit-For-Tat": 1,
		"Suspicious Tit-For-Tat": 1,
		"Tit-For-Two-Tats": 1
	}
\end{lstlisting}
\end{center}
 
\textbf{Configurația turneului cu eliminare: }

În turneul cu eliminare au participat câte 25 de indivizi, dintre care 5 sunt copii după cromozomul obținut de algoritmul genetic. 

- procentul de jucători care se elimină (și, implicit, procentul de jucători care se duplică) este de 25\%;\\
- configurația populației de testare:\\
\begin{center}
	\begin{lstlisting}
	{
		"Always Cooperate": 5,
		"Always Defect": 5,
		"Grudger": 5,
		"Pavlov": 5
	}
	\end{lstlisting}
\end{center}

\clearpage