Ne interesează să găsim \textit{cea mai bună strategie}. Pentru a compara între ele mai multe strategii, trebuie să gândim \textit{un mediu în care acestea să concureze}. 
 
Până la a găsi cea mai bună strategie, trebuie, mai întâi, să vedem cum anume poate configurația unui algoritm genetic să influențeze calitatea soluției. De asemenea, vrem să vedem cum anume se comportă într-un mediu de test strategia propusă de algoritm. 
 
Pentru a îndeplini aceste cerințe, am ales să supun cromozomii la un anumit tip de turneu, ce poartă denumirea de \textbf{turneu cu eliminare}.  

\section {Termeni întâlniţi}
O \textbf{rundă} este dată de alegerea, în mod secret, a mișcării următoare și actualizarea scorului în funcție de ce a pus și oponentul. 

Un \textbf{meci} este jucat de către doi jucători. Este alcătuit dintr-un număr de runde. În fiecare rundă, fiecare jucător alege, în mod secret, ce mișcare va face. La final de rundă, scorul jucătorilor este actualizat cu o valoare dată de mișcarea făcută de fiecare, în funcție de ce a ales și oponentul să facă. 

\section {Cum este modelat un turneu cu eliminare}
 
Un turneu cu eliminare pornește de la o populație de strategii în care, la fiecare iterație, fiecare individ joacă câte un meci cu ceilalți indivizi. Pe parcursul meciurilor, câștigurile individuale se însumează într-un scor total. După ce se joacă toate combinările de doi jucători (se ajunge la finalul iterației), se elimină un procent din cei mai slabi jucători.  Pentru a mai reduce din numărul parametrilor ale căror valori pot varia, am stabilit că procentul să fie de 25\%. În caz de egalitate a scorurilor între doi jucători, se elimină la întâmplare unul din cei doi. Se completează locurile eliberate cu stategii care au obținut printre cele mai bune scoruri. Se resetează scorul total al indivizilor și se repetă acești pași până când în turneu a rămas un singur tip de strategie, ori până când am atins un număr maxim de iterații, pe care l-am lăsat la valoarea de 10.  
\\\\
\textit{Observație}: Când într-un turneu cu eliminare concurează doi indivizi ce au aceeași strategie deterministă, la finalul unei iterații, cei doi indivizi vor avea exact același scor. Nu putem spune același lucru despre doi indivizi care folosesc strategia \textbf{Random}.
\\\\
Pentru a vedea clar modul în care evoluează strategiile în contextul acestui tip de turneu, am implementat o metodă grafică de vizualizare a datelor. Am ales să folosesc \textbf{line chart}-uri. Axa absciselor are drept legendă numărul de indivizi din fiecare strategie. Axa ordonatelor reprezintă indexul rundei turneului.

În fiecare turneu cu eliminare pot participa 25 de jucători, câte 5 din fiecare strategie. La fiecare turneu cu eliminare sunt câte 5 jucători care urmează exact aceeași strategie obținută în urma rulării algoritmului genetic.

\section {Concluzii trase in urma finalizarii turneelor}

În contextul acestei probleme, nu putem vorbi despre optim local sau global, întrucât o strategie care are un fitness bun în faza de antrenare poate să își piardă relevanța în faza de testare.  

\subsection{Numărul de runde al meciurilor din turneul cu eliminare}

Când numărul de runde al merciurilor din turneul cu eliminare este mic, am observat că strategiile propuse de algoritmul genetic, cu mici excepții, nu reușesc să câștige. Numărul de cromozomi, în cele mai optimiste cazuri, înregistrează o creștere și apoi o descreștere până la 0 la final de turneu. 

În exemplele de mai jos, configurația algoritmului genetic este următoarea: 

- numărul de runde jucate în fiecare meci: 20\\
- numărul de generații: 500\\
- dimensiunea populației antrenate: 25\\
- probabilitatea de încrucișare: 0.1\\
- probabilitatea de mutație: 0.2\\
- configurația populației de antrenament:\\
\begin{center}
	\begin{lstlisting}
	{
		"Always Cooperate": 1,
		"Always Defect": 1,
		"Grudger": 1,
		"Pavlov": 1,
		"Tit-For-Tat": 1,
		"Suspicious Tit-For-Tat": 1,
		"Tit-For-Two-Tats": 1
	}
\end{lstlisting}
\end{center}

\begin{figure}[H]
	\centering
	\includegraphics[	
		width=16cm,
		height=7.5cm,
		keepaspectratio
	]{imagini/chart_from_epoch_1528885395051.png}
	\caption{\textit{Evoluția turneului cu eliminare când numărul de runde este 5.}}
	\label{fig:evolutia_cand_numarul_de_runde_este_5}
\end{figure}

\begin{figure}[H]
	\centering
	\includegraphics[	
		width=16cm,
		height=7.5cm,
		keepaspectratio
	]{imagini/chart_from_epoch_1528885889300.png}
	\caption{\textit{Evoluția turneului cu eliminare când numărul de runde este 15.}}
	\label{fig:evolutia_cand_numarul_de_runde_este_15}
\end{figure}

Mărind cu 2 numărul de runde din fiecare meci, observăm că strategia propusă de algoritmul genetic câștigă. 

\begin{figure}[H]
	\centering
	\includegraphics[	
		width=16cm,
		height=7.5cm,
		keepaspectratio
	]{imagini/chart_from_epoch_1528886329612.png}
	\caption{\textit{Evoluția turneului cu eliminare când numărul de runde este 17.}}
	\label{fig:evolutia_cand_numarul_de_runde_este_17}
\end{figure}

Dacă numărul de runde al turneului cu eliminare este mai mare și nu este prezentă strategia \textbf{Random}, strategia algoritmului genetic are șanse mai mari de câștig.

\subsection{Relevanța populației de antrenament când populația de testare e diversificată}

Consider că \textbf{o populație diversificată de antrenament} poate simula condițiile din faza de testare. Din acest motiv am optat pentru ca aceasta să cuprindă fiecare strategie deterministă prezentată la secțiunea de strategii standard (am omis strategia \textbf{Random}). Prezența strategiei \textbf{Random} în faza de antrenament nu ajută la interpretarea calității soluției. Sunt șanse ca acestă strategie nedeterministă să obțină un scor bun în faza de antrenament, însă să nu facă față în faza de testare. 

Am pus câte un individ din fiecare strategie.

Am încercat să antrenez strategiile în contextul în care populația de antrenament era alcătuită doar din \textbf{Always Cooperate}, sau doar din \textbf{Always Defect}, sau doar din Tit-For-Tat, însă aceste strategii nu s-au comportat într-un mod notabil în faza de testare cand .

Un lucru ce se observă ușor este că strategia \textbf{Tit-For-Tat} reușește să câștige aproape de fiecare dată când la initiaizarea turneului cu eliminare toate strategiile au ponderi egale si numarul de runde este mare.

Strategia Tit-For-Tat poate totuși fi totuși bătută. Un astfel de context presupune ca strategiile cu care concurează să aleagă ca la fiecare pas să repete aceeași mișcare, asemeni strategiilor Always Defect și Always Cooperate. Motivul eliminării ar fi că la scoruri egale, când se dorește eliminarea unui singur jucător din 2 candidați, alegerea se face la întâmplare. Acest scenariu presupune schimbarea structurii populației de test. Observația este prezentată în exemplul de mai jos: 

\begin{figure}[H]
	\centering
	\includegraphics[	
		width=16cm,
		height=7.5cm,
		keepaspectratio
	]{imagini/chart_from_epoch_1528914497347.png}
	\caption{\textit{Strategia \textbf{Tit-For-Tat} pierde.}}
	\label{fig:tit_for_tat_pierde}
\end{figure}

Dacă vrem ca în exemplu să includem un cromozom și facem astfel încât strategia \textbf{Tit-For-Tat} să câștige, trebuie să defavorizam cromozomul lăsând numărul de runde la o valoare mică, cum ar fi 5. Dacă ar fi o valoare mai mare, în prima instanță s-ar elimina strategia care repetă aceeași mișcare, ca mai apoi să fie eliminate copiile cromozomilor. 

\subsection{Strategia \textbf{Random}}
 
Am reușit să găsesc o soluție pentru a bate strategia \textbf{Random}. În cel mai rău caz s-a ajuns la remiză. Am ales o populație cu dimensiuni mici (15), un număr de generații mare (1000), valori mari și pentru mutație și încrucișare (în jurul valorii de 50\% pentru fiecare dintre cele două). Am ales ca populația de antrenament să fie alcătuită din copii ale strategiei Random. Numarul de runde l-am fixat la valoarea de 100.








































































































































































































































