\chapter*{Introducere}

Studiul teoriei jocurilor se resimte în multe domenii. Spre exemplu, în cloud computing, teoria jocurilor poate fi aplicată pentru a modela interacțiunile dintre diverși provideri de servicii cloud, unde scopul clientului este minimizarea costurilor și a pierderilor și maximizarea utilizării resurselor. În domeniul afacerilor, în mod constant, se iau decizii legate de menținerea sau reducerea costurilor unui anumit produs, de crearea unor noi produse, de retragerea lor de pe piață. Aplicând elemente de teoria jocurilor, se pot dezvolta strategii, se pot prezice tendințe și se pot face presupuneri despre modul în care anumite decizii pot influența succesul unei campanii. 

Dilema prizonierului este o problemă analizată în teoria jocurilor în care participanții trebuie să aleagă, în mod secret, între a coopera cu celalalt protagonist și a-l trăda și, cu toate că ar avea parte de cel mai mare câștig dacă ar alege cooperarea, neîncrederea în oponent poate duce la trădare.  Dilema prizonierului este exemplificată în rândurile următoare. 
 
Două firme competitive pun la vânzare același produs. Presupunem că, la un anumit moment, cele două firme iau în calcul, în mod independent, schimbarea prețului produsului. Una dintre firme dorește să obtină un profit mai mare, așa că reduce prețul produsului, lucru ce poate fi interpetat drept trădarea competitorului. Celălalt întreprinzător alege între a coopera menținerea prețul și a trăda prin scăderea prețului, egalând prețul propus de competitor sau oferind un preț și mai mic. O altă variantă ar fi ca cele două firme să lase produsul la același preț. Acest comportament al agenților economici ar însemna împărțirea pieței între aceștia. 
  
Varianta iterată a dilemei prizonierului produce strategii care rezumă tendințele legate de încrederea jucătorilor.  

\chapter*{Contribuții}

În prezenta lucrare propun studierea stategiilor obținute cu ajutorul algoritmilor genetici, introducându-le într-un mediu în care populația de jucători este într-o continuă schimbare. În acest scop, am desfășurat o serie de experimente. Cadrul de testare al soluțiilor algoritmului reprezintă un turneu în care, după ce fiecare jucător joacă cu toți ceilalți participanți un meci, o parte dintre aceștia sunt eliminați și o parte duplicați. E o analogie a faptului că un comportament care duce la succes va fi imitat și unul care duce la eșec își va pierde din popularitate.  Odată aleasă o anumită strategie, jucătorul nu o poate modifica în mijlocul turneului. Turneul se încheie atunci când populație folosește aceeași strategie. 
 
Acest mediu reprezintă un simplu joc, insă, prin exemplul pe care îl oferă, deducem că putem modela noi medii, care să respecte criteriile impuse de problema cu care ne confruntăm. În acele medii putem introduce interacțiuni între indivizi care iau decizii ținând cont de tendința celorlalți participanți de a coopera sau trăda.

Pentru a trage cele mai bune concluzii, trebuie să facem un studiu prin care să identificăm strategiile adoptate de actorii din problema noastră, să modelăm un mediu de antrenament cu ajutorul unui algoritm genetic și un mediu de testare. În cadrul unor experimente, diverse soluții ale algoritmului genetic vor fi supuse la test, rezultatele putând fi, ulterior, interpretate și propuse pentru rezolvarea problemei.  


