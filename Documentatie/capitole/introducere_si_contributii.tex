\chapter*{Introducere}

Studiul teoriei jocurilor se resimte în multe domenii. Spre exemplu, în cloud computing, teoria jocurilor poate fi aplicată pentru a modela interacțiunile dintre diverși provideri de servicii cloud, unde scopul clientului este minimizarea costurilor și a pierderilor și maximizarea utilizării resurselor. În domeniul afacerilor, în mod constant, se iau decizii legate de menținerea sau reducerea costurilor unui anumit produs, de crearea unor noi produse, de retragerea lor de pe piață. Aplicând elemente de teoria jocurilor, se pot dezvolta strategii, se pot prezice tendințe și se pot face presupuneri despre modul în care anumite decizii pot influența succesul unei campanii. 

Dilema prizonierului este o problemă analizată în teoria jocurilor în care participanții trebuie să aleagă, în mod secret, între a coopera cu celalalt protagonist și a-l trăda și, cu toate că ar avea parte de cel mai mare câștig dacă ar alege cooperarea, neîncrederea în oponent poate duce la trădare.  Dilema prizonierului este exemplificată în rândurile următoare. 
 
Două firme competitive pun la vânzare același produs. Presupunem că, la un anumit moment, cele două firme iau în calcul, în mod independent, schimbarea prețului produsului. Una dintre firme dorește să obtină un profit mai mare, așa că reduce prețul produsului, lucru ce poate fi interpetat drept trădarea competitorului. Celălalt întreprinzător alege între a coopera menținerea prețul și a trăda prin scăderea prețului, egalând prețul propus de competitor sau oferind un preț și mai mic. O altă variantă ar fi ca cele două firme să lase produsul la același preț. Acest comportament al agenților economici ar însemna împărțirea pieței între aceștia. 
  
Varianta iterată a dilemei prizonierului produce strategii care rezumă tendințele legate de încrederea jucătorilor.  

\chapter*{Contribuții}