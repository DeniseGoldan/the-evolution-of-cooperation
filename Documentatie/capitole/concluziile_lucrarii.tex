\chapter*{Concluziile lucrării}

De la începutul proiectului am avut o bănuială cum că nu putem vorbi despre existența unei strategii optime. Această idee este susținută de faptul că utilizatorul poate configura mediul de testare după bunul plac. Schimbarea unui singur parametru poate însemna diferența dintre succes și eșec, așa cum s-a putu observa în experimentul cu numărul 2 (\textit{Numărul de runde al meciurilor din turneul cu eliminare}). După o serie de experimente, dintre care o parte a fost legată în această lucrare, bănuiala mi-a fost confirmată.  
 
Chiar dacă nu există o strategie care să domine turneul cu eliminare în orice formă a sa, putem folosi algoritmul genetic pentru a modela soluția care să fie optimizată pentru un mediu de testare cunoscut (așa cum s-a putut observa în experimentele desfășurate cu scopul de a învinge strategia \textbf{Random} și strategia \textbf{Tit-For-Tat}). 
  
Cred că această lucrare reprezintă un exemplu ce atestă că putem căuta soluții pentru diverse probleme, apelând la tehnicile folosite aici. Dată fiind o problemă, putem să identificăm actorii, strategiile folosite de aceștia legat de tendințele de cooperare și trădare și putem experimenta cu diverse strategii propuse de o euristică. Euristica favorizează căutarea unei soluții într-un spațiu foarte mare de căutare. Pentru găsirea unei soluții cât mai bune, care să se potrivească mediului de testare, trebuie să derulăm un număr mare de experimente și trebuie să analizăm rezultatele. 

\clearpage