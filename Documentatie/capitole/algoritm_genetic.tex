\section{Aparitia notiunii de algoritm genetic}

Algorimii genetici\footnote{Adaptare dupa \textit{"Genetic Algorithms: An Overview"}, Melanie Mitchell.} au fost introdusi de catre John Holland in 1960 si dezvoltati, ulterior, alaturi de colegii de la Universitatea din Michigan, intre anii 1960 si 1970. Holland urmarea intelegerea fenomenului de "adaptare" intalnit in natura si implementarea unor mecanisme adaptive care sa fie utilizate in practica, in contextul programarii. Cartea publicata de acesta in 1975, \textit{"Adaptation in Natural and Artificial Systems"} (Holland, 1975/1992) prezinta algoritmii genetici drept abstractizari ale evolutiei biologice, si ofera un cadru teoretic pentru dezvoltarea acestora. Algoritmii genetici ai lui Holland sunt metode de a trece de la o populatie de "cromozomi" (e.g., siruri de "biti" care reprezinta solutii candidat pentu o problema) la o noua populatie, prin folosirea "selectiei", alaturi de operatorii insiprati din genetica: incrucisare, mutatie, inversiune. Cea din urma este rar folosita in practica.

\begin{quote} 
	\textit{"Computer programs that "evolve" in ways that resemble natural selection can solve complex problems even their creators do not fully understand."}
	\begin{flushright}
		by John H. Holland 
	\end{flushright}
\end{quote}

Algoritmii genetici au fost creati in incercarea de a imita procese specifice evolutiei naturale, cum ar fi lupta pentru supravietuire si mostenirea materialului genetic. Putem privi evolutia drept strategia abordata de speciile biologice pentru a cauta "solutii" cat mai potrivite, adaptate conditiilor schimbatoare, intr-un numar foarte mare de posibilitati. Aceasta abordare poate fi utilizata in rezolvarea problemelor de optimizare, atunci cand metodele clasice exhaustive nu se dovedesc eficiente.

Notiunea de "algoritm genetic" nu este definita in mod riguros\cite{introduction_by_melanie_mitchell}, insa toate metodele ce poarta aceasta denumire au in comun urmatoarele: populatia este formata din cromozomi, selectiz este facuta pe baza rezultatelor functiei de optimizat, incrucisarea a doi \textit{cromozomi parinti} produce 2 \textit{cromozomi copii}, mutatia se aplica \textit{cromozomilor copii}.

\section{Terminologie}

\begin{itemize}
	
	\item Solutiile candidat sunt adesea codificate in forma unor siruri de biti si se mai numesc \textbf{cromozomi} sau \textbf{indivizi} ai populatiei. Fiecare bit este echivalentul unei gene.
	
	\item Genele sunt informatiile stocate de catre cromozomi.
	
	\item \textbf{Populatia}, care va fi urmarita in procesul sau evolutiv, este alcatuita din mai multi cromozomi.
	
	\item Fiecare \textbf{generatie} marcheaza cate o etapa din evolutia populatiei initiale.
	
	\item Pentru a trece de la o generatie la alta, apelam la notiunea de \textbf{reproducere}. In alcatuirea urmatoarei generatii, se porneste de la populatia actuala, pe care o supunem unui proces de \textbf{selectie}. Pentru a face analogia cu fenomenul de supravietuire a celor mai adapati indivizi, masuram cromozomii cu ajutorul unei functii de optimizat. O valoare ridicata a acestei functii este interpretata ca o buna adaptare la mediu a individului. 
	
	\item Pentru explorarea spatiului de solutii, indivizii selectati sufera modificari. Sunt supusi \textbf{incrucisarilor} si \textbf{mutatiilor}.
	
	\item Incrucisarea combina genele a doi \textit{cromozomi parinti}, rezultand doi \textbf{mostenitori}. Exista mai multe variante: cu un punct de taiere, ales aleator (in care un mostenitor este alcatuit dintr-o portiune de cromozom de la primul parinte si o portiune de la al doilea), cu mai multe puncte de taiere si uniforma (unde fiecare gena este selectata probabilit de la unul din cei 2 \textit{cromozomi parinti}).
	
	\item Mutatia altereaza gene alese arbitrar dintr-un cromozom. Numarul de gene afectate poate varia.
	
\end{itemize}

\section{Pseudocod}

\begin{verbatim}
initializeaza cu valori aleatorii populatia initiala
calculeaza valoarea functiei de optimizat pentru indivizii populatiei 
cat timp nu s-a indeplinit conditia de oprire
    aplica o metoda de selectie, pentru a crea populatia
    aplica operatorul genetic incrucisare, cu o anumita probabilitate
    aplica operatorul genetic mutaatie, cu o anumita probabilitate
calculeaza valoarea functiei de optimizat pentru indivizii populatiei
\end{verbatim}

Conditia de oprire poate fi atingerea unui numar de iteratii stabilit initial. De asemenea, se poate stabili ca algoritmul sa se opreasca atunci cand nu se mai inregistreaza imbunatatiri in ceea ce priveste calitatea solutiilor furnizate. 

Solutia returnata de un algoritm genetic reprezinta cel mai bun individ intalnit in evolutia populatiei. 

\hfill \break
\hfill \break

\clearpage