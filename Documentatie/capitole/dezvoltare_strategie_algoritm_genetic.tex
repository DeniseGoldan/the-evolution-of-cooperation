\section{Modelarea unui algoritm genetic pentru problema iterată a prizonierului}

\subsection{Reprezentarea soluției}

În urma organizării celor două turnee, Axelrod a decis să cerceteze dezvoltarea unei strategii pentru problema iterata a prizonierului, folosindu-se de algoritmii genetici introduși de Holland. 

Unul din cei mai importanți pași din acest proces a fost stabilirea unei modalități de a reprezenta o strategie în forma unui cromozom. Descriem concluziile lui Axelrod în rândurile următoare.

Presupunem că fiecare jucător are capacitatea de a memora mișcările ultimei runde sub forma unei perechi- primul element reprezintă mișcarea proprie, iar cel de al doilea element reprezintă mișcarea oponentului. Ne vom folosi de următoarea notație:

\begin{center}
	\textbf{C} reprezintă \textbf{cooperarea cu oponentul}\\
	\textbf{T} reprezintă \textbf{trădarea oponentului}   
\end{center}

Există patru perechi (sau cazuri) posibile:\\

\begin{center}
	Cazul 1: \textbf{CC}\\
	Cazul 2: \textbf{CT}\\
	Cazul 3: \textbf{TC}\\
	Cazul 4: \textbf{TT}\\
\end{center}

Pentru acest scenariu, strategia reprezintă ce mișcare vom alege, dat fiind aflarea mișcării oponentului în ultima rundă.

Strategia \textbf{Tit-for-Tat} este reprezentată în felul următor: 

\begin{center}
	dacă \textbf{CC} atunci \textbf{C}\\
	dacă \textbf{CT} atunci \textbf{T}\\
	dacă \textbf{TC} atunci \textbf{C}\\
	dacă \textbf{TT} atunci \textbf{T}\\
\end{center}

Dacă impunem ca aceste patru cazuri să respecte ordinea lexicografică, putem codifica strategia drept șirul de caractere \textbf{CTCT}. Ca să utilizăm această reprezentare a strategiei:\\
\begin{itemize}
	\item Observăm ce a ales oponentul în runda anterioară;
	\item Formăm perechea compusă din mișcarea noastră, împreună cu cea a oponentului;
	\item Vedem indexul care corespunde perechii obținute la pasul anterior;
	\item Alegem mișcarea pe care o găsim la indexul respectiv.
\end{itemize}

Strategiile lui Axelrod se bazau pe istoricul ultimelor trei runde. Pentru acestea, există 64 \footnote{64, sau $2^6$, reprezintă numărul de șiruri de caractere unice pe care le putem genera folosind doar caracterele \textbf{C} și \textbf{T}} de posibile scenarii pentru utlimele trei runde: 

\begin{center}
	Cazul 1: \textbf{CC CC CC}\\
	Cazul 2: \textbf{CC CC CT}\\
	Cazul 3: \textbf{CC CC TC}\\
	...
	\\
	Cazul 62: \textbf{TT TT CT}\\
	Cazul 63: \textbf{TT TT TC}\\
	Cazul 64: \textbf{TT TT TT}\\
\end{center}

Ca și în  ipoteza în care jucătorii memorează doar istoricul ultimei runde, putem reprezenta aceste cazuri într-un șir de caractere de lungime 64. Vom folosi un șir de caractere de lungime 71, pentru a reține și ce mișcări ar trebui făcute în primele runde, când încă nu există un istoric care să cuprindă ultimele trei runde. Cele 7 poziții de la începutul șirului de caractere au următoarele semnificații:\\

\begin{enumerate}
	
	\item La \textbf{poziția numărul 1} se găsește mișcarea aleasă pentru prima rundă a jocului;
	
	\item \textbf{Poziția numărul 2}: mișcarea pentru cea de a doua rundă, dacă la rundă anterioară oponentul a cooperat (\textbf{C} reprezintă istoricul mișcărilor oponentului);
	
	\item \textbf{Poziția numărul 3}: mișcarea pe care o vom face la cea de a două rundă, dacă la rundă anterioară oponentul a trădat (istoricul mișcărilor oponentului este scris sub forma \textbf{T});
	
	\item \textbf{Poziția numărul 4}: mișcarea pe care o vom face la cea de a două rundă, dacă pentru primele două runde, istoricul oponentului este \textbf{CC};
	
	\item \textbf{Poziția numărul 5}: mișcarea pentru cea de a treia rundă, dacă pentru primele două runde, istoricul oponentului este \textbf{CT};
	
	\item \textbf{Poziția numărul 6}: mișcarea pentru cea de a treia rundă, dacă pentru primele două runde, istoricul oponentului este \textbf{TC};
	
	\item \textbf{Poziția numărul 7}: mișcarea pentru cea de a treia rundă, dacă la primele două runde oponentul a avut mișcările \textbf{TT}.
	
\end{enumerate}

\subsection{Dimensiunea spațiului de căutare}

Având 71 de poziții pe care le putem ocupa cu cele două caractere- \textbf{C} sau \textbf{T}-, putem genera $2^{71}$ șiruri de caractere distincte. Acest număr reprezintă numărul tuturor strategiilor pe care îl putem avea, în contextul în care cunoaștem istoricul ultimelor trei runde ale jocului.  

Spațiul de căutare este, în concluzie, mult prea mare pentru a caută exhaustiv cea mai bună strategie.

\subsection{Funcția de optimizat}

Axelrod a alcătuit un set de opt strategii cu care să concureze fiecare strategie a algoritmului genetic, în vederea calculării unei funcții de optimizat. Acest set de strategii nu include strategia \textbf{Tit-for-Tat}. Valoarea funcției de optimizat este dată de media scorurilor obținute în urma meciurilor jucate cu fiecare dintre ele opt strategii.

\subsection {Parametrii algoritmului genetic}

Prin utilizarea unui algoritm genetic, pot obține o copie a celui mai bun individ din toate generațiile care au participat la antrenare. În alte cuvinte, acest cromozom conține strategia care a obținut cea mai bună valoare a funcției de optimizat. 

Pentru crearea acestui individ, este nevoie de ajustarea mai multor parametri și opțiuni, dintre care menționez: ratamutației, rata încrucisării, tipul selecției populației (populația diferă ușor de la generație la generație, selectandu-se doar anumiți indivizi și în anumite proporții). Cromozomul are capacitatea de a-și formulă următoarea mișcare bazându-se pe istoricul ultimelor trei runde. Din acest motiv este important ca un meci să fie format din mai multe runde. Așadar, numărul de runde reprezintă și acesta un parametru pentru antrenarea cromozomilor. 

-- OBSERVAȚII CLARE OBȚINUTE ÎN URMĂ UNOR EXPERIMENTE legat de modul în care parametrii influențează calitatea soluțiilor

\section {Elemente legate de implementarea algoritmului genetic}

-- CUM AM IMPLEMENTAT EU ACEASTĂ PROBLEMA

-- UȘURINȚA CU CARE SE POATE CREA O NOUĂ STRATEGIE

-- UNIT TESTS

\section {Limitările algoritmului genetic}

- care sunt limitările unui algoritm genetic aplicat pe această problema?